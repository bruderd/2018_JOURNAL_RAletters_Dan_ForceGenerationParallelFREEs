\section{Introduction}
\label{sec:intro_draft}

Soft continuum manipulators, resembling biological trunks and tentacles, offer capabilities beyond the scope of traditional rigid link manipulators. Namely, their compliant structure allows them to adapt their shape to navigate unstructured environments, safely work alongside humans, and absorb large impacts without damage. A central feature of continuum manipulators is continuously deformable backbones in place of rigid links connected by discrete joints. In order to utilize this feature effectively, actuation methods are needed for continuum manipulators that do not inhibit their compliance.

Current manipulators utilize one of two actuation approaches, intrinsic or extrinsic. Intrinsically actuated manipulators have actuators mounted directly to the compliant backbone, producing forces locally. Soft fluid-driven actuators such as bellows (cite) or McKibbon muscles (cite) are popular choices for such schemes due to their ability to produce forces without imposing structure. For example, the McKibbon actuators used in the Octarm robot produce large contractile forces but little resistance to bending, enabling curvature along the length of the entire arm (cite). Extrinsically actuated manipulators have actuators located off of the manipulator arm itself, and transfer forces to the arm via cables (cite). The benefit of this approach is that it can rely on more conventional actuators such as motors, and cables can be fixed to the arm in arbitrary ways, producing both bending and twisting forces.

One downside of intrinsic actuation schemes is that the direction of the forces applied cannot be redirected arbitrarily, since actuators are mounted directly to the backbone. For this reason, most intrinsically actuated continuum manipulators are adept at producing bending moments, but not twisting. Extrinsic actuation schemes do better in this regard, as cables can be routed and attached to the backbone in ways that produce both bending and twisting moments (cite Clemson, surgical robots). However, cable driven schemes introduce complications such as slack and backlash, which typically require complex mechanical or control systems to remedy (cite).

The fiber reinforced elasomeric enclosure (FREE) is a type of fluid driven soft actuator that combines the simplicity of intrinsic actuation with the versatility of extrinsic actuation. Composed of an elastomeric tube wound with fibers, FREEs simultaneously exert forces along their central axis as well as moments about that axis (see Figure \ref{fig:FMratios}). This enables FREEs mounted directly to a compliant backbone to produce both bending and twisting moments, without introducing complications like slack or backlash. The ratio between the force and moment produced by a FREE is also customizable, determined by the angle of the wrapped fibers.

[[In this paper we introduce the concept of a force polygon [as the set of all possible forces that can be generated by a parallel combination of actuators] as a way of comparing the relative effectiveness of [parallel combinations] of different types of actuators in generating arbitrary forces. Namely, the dimension of the force zonotope corresponds to the number of actuatable degrees of freedom. We validate the static model presented by comparing its predictions to those measured on a physical system. We conclude with a discussion of the possible sources of error and possible future directions.]]


-----------------------------------------------------------------------------------------------------------

A common actuation approach for continuum manipulators is extrinsic actuation schemes. In such schemes actuators are located at the base of the robot and transfer forces to the manipulator via cables, or ``tendons''. The tendons exert forces only in the direction of pull, which allows them to transfer large forces without imposing rigidity on the overall structure of the robot. A weakness of extrinsic actuation is the need to account for slack and backlash in the tendons (cite), which can introduce unwanted complexity into the system.

Alternatively, intrinsic actuation schemes avoid the problems of slack and backlash by mounting actuators directly to the manipulator. Soft fluid-driven actuators such as bellows (cite) or McKibbon muscles (cite) are popular choices for such schemes due to their ability to produce forces without imposing structure. For example, the McKibbon actuators used in the Octarm robot produce large contractile forces but little resistance to bending, enabling curvature along the length of the entire arm (cite).

The shortcoming of both actuation is that they can produce bending forces but not twisting forces


A common feature of each actuation scheme mentioned is deformability in all directions other than that of actuation


In principle such continuously deformable structures have kinematics that are infinite dimensional, but their actuation schemes are necessarily finite dimensional. Actuation schemes that enable control of some degrees of freedom without significantly constraining the others is desirable, since inherent compliance is what we want.


--------------------------------------------------------

Soft robots require actuation methods that compliment (or do not inhibit) their compliant structure. For this reason, fluid driven actuators composed of soft deformable materials have become popular for such applications (cite, cite, cite, ...). Despite their apparent benefits, soft actuators have yet to be widely adopted by the robotics community, due to a lack of understanding of how they generate forces. While some work has been done to characterize soft actuators (cite McKibbons, bellows, FREEs), their nonlinear nature make them less attractive than conventional means of actuation... 

due to the low cost of electric motors and well developed theory for using them as actuators... While the benefits of using soft actuators is apparent, characterizing the force generation properties of such actuators remains a challenge.

%Some actuators are made to generate forces, others are made to generate torques.
%Fiber reinforced elastomeric enclosures (FREEs) have a mechanically programmible force/torque ratio, which has benefits for the dexterity of parallel soft manipulators.

The most common type of actuator in robotic systems is the electric motor, but motors are not well suited to directly drive soft robots due to their rigidity. Therefore, to actuate soft robots, motors must be used indirectly via cable drive systems (cite Clemson, Walsh exo, surgical robots, etc.). Being located so far away from the point of use introduces friction, binding, and `slack', all of which decrease the effectiveness of force generation at the point of use (cite?). Therefore, fluid driven actuators like bellows (cite) or McKibbons (cite) are commonly used in soft robots instead of motors. Fluid driven systems still require bulky components such as valves and pumps, but these components may be located remotely where they do not interfere with a robot's compliant structure, and forces can be applied directly at the point of use (please replace `point of use' with something more clear*).

Actuators generate forces and moments to make robots move. In general, actuators either generate forces along a central axis (e.g. pistons, skeletal muscle), or they generate moments about an axis (e.g. motors). Most fluid driven actuators are of the first type, converting fluid pressure into a force along their actuation direction. 

Unlike motors, fluid driven actuators can only generate forces in one direction, so they are often combined in antagonistic pairs. To fully actuate a joint with 6 degrees of freedom this way, 12 separate actuators are needed (this is how our bodies are actuated). Each of those actuators requires some sort of bulky valve or compressor to drive it, so we really want to have as few actuators as possible.

Forces can then be converted into rotational moments or vice versa via a transmission, with some power lost to friction, heat, etc. in the process. It is often desirable to have as many controllable DOFs as possible (i.e. robotic wrists...)





One such fluid driven actuator is a FREE, which is unique in that it has a customizable force/moment ratio based on fiber angle.

[[In this paper we introduce the concept of a force zonotope [as the set of all possible forces that can be generated by a parallel combination of actuators] as a way of comparing the relative effectiveness of [parallel combinations] of different types of actuators in generating arbitrary forces. Namely, the dimension of the force zonotope corresponds to the number of actuatable degrees of freedom, [and the volume gives a sense of the magnitude of forces generated]. We validate the static model presented by comparing its predictions to those measured on a physical system. We conclude with a discussion of the possible sources of error and possible future directions.]]

\subsection{BASIC ARGUMENT FOR FREEs}

The programmable force/torque ratio of FREEs can be exploited/leveraged/utilized in parallel combinations to generate forces/torques in more directions than possible with pure force/torque sources, without need of any sort of transmission. A FREE consists of an elastomeric tube with fibers wound in helices around it. In this paper we consider only FREEs with a single fiber family, i.e. all fibers are wound in parallel helices.

\begin{figure}
\centering

\begin{tikzpicture}

% Define global variables of the tikzpicture:
\def\scl{0.37}

\matrix[row sep=0cm, column sep=0.3cm, style={align=center}] (my matrix) at (0,0)
{
\begin{axis}[
    axis lines=center,
    xlabel = {$F$}, xmin=-3, xmax=3, xtick={0}, xlabel style={anchor=north east},
    ylabel = {$M$}, ymin=-3, ymax=3, ytick={0}, ylabel style={anchor=east},
    scale=0\scl,
    anchor=center,
    ]
    \addplot [->, line width=1.5pt, blue] coordinates {(0,0) (0,2)};
\end{axis};
&
\begin{axis}[
    axis lines=center,
    xlabel = {$F$}, xmin=-3, xmax=3, xtick={0}, xlabel style={anchor=north east},
    ylabel = {$M$}, ymin=-3, ymax=3, ytick={0}, ylabel style={anchor=east},
    scale=\scl,
    anchor=center,
    ]
    \addplot [->, line width=1.5pt, blue] coordinates {(0,0) (-2,0)};
\end{axis};
&
\begin{axis}[
    axis lines=center,
    xlabel = {$F$}, xmin=-3, xmax=3, xtick={0}, xlabel style={anchor=north east},
    ylabel = {$M$}, ymin=-3, ymax=3, ytick={0}, ylabel style={anchor=east},
    scale=\scl,
    anchor=center,
    ]
    \addplot [->, line width=1.5pt, blue] coordinates {(0,0) (-1,1.7)};
\end{axis};
\\
\node[style={anchor=center}] {(a)};
& 
\node[style={anchor=center}] {(b)};
& 
\node[style={anchor=center}] {(c)};
\\
};
\end{tikzpicture}
    
\caption{The force/torque combination for a (a) conventional torque source (e.g. motor), (b) conventional force source (e.g. piston), (c) combined force and torque source (e.g. FREE). The fiber reinforced actuator has a customizable force/torque ratio, unlike most conventional actuators.}
\label{fig:FMsingle}
\end{figure}










%Hello does this work \citet{bruder2017}

% Cable drives have also been used (Clemson robot, Walsh Exo) since cables buckle in all directions except the direction of pull

% Others (cite, cite, cite) have discovered that by coupling inextensible materials and loading them in tension you can do cool stuff...

% In general, actuators come in two flavors: torque source (i.e. motors) and force source (i.e. pistons, bellow, McKibbon). The force/torque of one of these sources can be converted to the other via a transmission, with some power lost to friction, heat, etc. (i.e. crank-cam shaft, pulley, worm drive, etc..).

% motors pulling on cables, as well as other fluid...

% The most common actuator in robotic systems is the electric motor (cite reasons why... low cost, controllable, etc.). Electric motors generate rotational moments of force, but with the aid of a transmission this moment can be converted into a linear force (e.g. worm drive linear actuator). Hence they are ubiquitous in all types of robotic systems. However, electric motors are not suitable actuators for soft robots because they are rigid in nature, and typically require gearing... (this makes them bulky). Therefore, fluid driven actuators like bellows (cite) or McKibbons (cite) are commonly used in soft robots instead. Fluid driven systems still require bulky components such as valves and pumps, but these components may be located remotely where they do not interfere with a robot's compliant structure. In contrast, electric motors must typically be located close to or directly on the motion axis (hold on this isn't true, what about pulleys?...).


%%%%% SAVE AND REUSE THIS TEXT FOR INTRO TO CONTINUUM MANIPULATOR PAPER, FOR NOW IT DOESN'T FIT THE FORCE GENERATION FOCUS OF THIS PAPER
% A defining characteristic of soft continuum robots is an attribute known as ``inherent compliance'', which means that their shape conforms both to constraints applied by the actuators as well as constraints imposed by the environment. This attribute arises from the fact that continuum manipulators have kinematics that are infinite-dimensional, but actuation schemes that are necessarily finite-dimensional. Nature utilizes inherent compliance to create manipulators that are adept at navigating unknown environments (duck penises and such...) and manipulate irregular objects (octopus tentacles, elephant trunks). Roboticists, too, are leveraging inherent compliance to build soft continuum robots that are safer and more dexterous than their rigid bodied counterparts (cite clemson, octarm, MIT, and any other soft continuum manipulator). 